\documentclass{article}

\usepackage[utf8]{inputenc}
\usepackage[T1]{fontenc}
\usepackage[francais]{babel}
%Chargement des packages
\usepackage{graphicx}
\usepackage{ifthen}
\usepackage{amsmath}
\usepackage{multicol}
\usepackage{multirow}
\usepackage{amsfonts}
\usepackage{tabulary}
\usepackage{listings}
\usepackage{xspace}
\usepackage{amssymb}
\usepackage{eurosym}
\usepackage[dvipsnames]{xcolor}
\usepackage[pdfborder={0 0 0}]{hyperref}
\usepackage[top=2cm, bottom=2cm, left=2cm, right=2cm]{geometry}
\lstset{literate=
  {á}{{\'a}}1 {é}{{\'e}}1 {í}{{\'i}}1 {ó}{{\'o}}1 {ú}{{\'u}}1
  {Á}{{\'A}}1 {É}{{\'E}}1 {Í}{{\'I}}1 {Ó}{{\'O}}1 {Ú}{{\'U}}1
  {à}{{\`a}}1 {è}{{\`e}}1 {ì}{{\`i}}1 {ò}{{\`o}}1 {ù}{{\`u}}1
  {À}{{\`A}}1 {È}{{\'E}}1 {Ì}{{\`I}}1 {Ò}{{\`O}}1 {Ù}{{\`U}}1
  {ä}{{\"a}}1 {ë}{{\"e}}1 {ï}{{\"i}}1 {ö}{{\"o}}1 {ü}{{\"u}}1
  {Ä}{{\"A}}1 {Ë}{{\"E}}1 {Ï}{{\"I}}1 {Ö}{{\"O}}1 {Ü}{{\"U}}1
  {â}{{\^a}}1 {ê}{{\^e}}1 {î}{{\^i}}1 {ô}{{\^o}}1 {û}{{\^u}}1
  {Â}{{\^A}}1 {Ê}{{\^E}}1 {Î}{{\^I}}1 {Ô}{{\^O}}1 {Û}{{\^U}}1
  {œ}{{\oe}}1 {Œ}{{\OE}}1 {æ}{{\ae}}1 {Æ}{{\AE}}1 {ß}{{\ss}}1
  {ű}{{\H{u}}}1 {Ű}{{\H{U}}}1 {ő}{{\H{o}}}1 {Ő}{{\H{O}}}1
  {ç}{{\c c}}1 {Ç}{{\c C}}1 {ø}{{\o}}1 {å}{{\r a}}1 {Å}{{\r A}}1
  {€}{{\euro}}1
}
\lstdefinelanguage{cisco}{
  keywords={no},
  ndkeywords={Loopback,Serial,FastEthernet,Ethernet},
  sensitive=false,
  comment=[l]{!},
  morecomment=[s][\color{PineGreen}]{[}{]},
}
\lstset{language=cisco}
%paramètrage
\lstset{basicstyle=\ttfamily\color{darkgray},
		numbers=left,
		frame=single,
		breaklines=true,
		stringstyle=\color{PineGreen},
		commentstyle=\color{Tan},
		keywordstyle=\bfseries\color{RedOrange},
		inputencoding=utf8,
    	extendedchars=true}

\title{Mission Telredor-Supra : Dossier de configuration}
\date{3 Juin 2019}
\author{Clément \bsc{Boutin} \and Baptiste \bsc{Saclier}}

\newcommand{\tlr}{Telredor\xspace}
\newcommand{\spr}{Supra Télécom\xspace}

\newcommand{\seefile}[1]{
  \begin{center}
  \begin{minipage}{0.9\textwidth}
    \emph{Exemple}: voir \texttt{\href{https://github.com/EpicKiwi/Wide-Network-Project-Cesi-A4/blob/master/network/#1}{#1}}
  \end{minipage}
  \end{center}
}

\begin{document}

\maketitle

\begin{center}
\rule{0.5\textwidth}{0.4pt}
\end{center}

\tableofcontents

\section{Introduction}

Ce document est un receuil de configuration diverses de l'ensemble de la mission d'harmonisation du réseau de \tlr et de \spr.
L'ensemble des configuration de la maquette sont disponibles à l'adresse \url{https://github.com/EpicKiwi/Wide-Network-Project-Cesi-A4/tree/master/network}.

\medskip

La maquette donne une idée du réseau et donne les modèles de configuration pour le déploiement sur site.
Il est aisé de déployer un système de redondance en ajustant la configuration sur sur la base des informations contenues dans ce document et dans le \emph{Dossier récapitulatif} pour garantir un meilleur SLA.

\paragraph{Nomenclature} Les modèles de configuration de ce document contiennent des description entourés de crochets qui doivent être remplacés par les informations nécéssaires.
Exemple : \texttt{\color{PineGreen}[Adresse interne client]} soit être remplacé par l'adresse IP interne du client.

\section{Modèle de configuration \spr}

Les routeurs \spr composent le nuage par lequel les clients de l'entreprise accèdent à un réseau multisites.
Les routeurs de ce nuage utilisent les protocoles OSPF pour le routage interne, MPLS pour le switching de paquets des clients et BGP pour le routage entre les VRF des clients.
Pour plus d'informations sur les protocoles utilisés, veuillez consulter le \emph{Dossier récapitulatif}.

Les routeurs du nuage sont tous basés sur un même modèle de configuration qui peut être réutilisé dans une optique d'élargissement.

\begin{lstlisting}[caption=Modèle de configuration d'un routeur nuagique]
enable
conf t

! Nom unique du routeur sur le nuage Supra
hostname [Nom routeur]

! Mise en place de la communication MPLS et des communications de labels
mpls ldp advertise-labels
! L'interface de Loopback 0 servira d'identifiant du routeur
mpls ldp router-id Loopback 0 force

! Déclaration des VR client et de leur RD assocés
ip vrf [Nom VRF client]
  rd 65000:1
  route-target both 65000:1

! Interface servant d'identifiant au travers de son adresse IP
interface Loopback 0
  ip address [Adresse IP d'identification routeur] 255.255.255.255

! Interface de connexion à un autre routeur du nuage
interface Serial 0/0
  encapsulation ppp
  ip address [Adresse IP routeur] 255.255.255.252
  mpls ip
  no shutdown

! Interface de connexion à un client sur une VRF particulière
interface Serial 0/3
  encapsulation ppp
  ip vrf forwarding [Nom VRF client]
  ip address [Adresse interne client] [Masque interne client]
  no shutdown

! Association des interfaces à la zone 0 globale du nuage Supra
router ospf 1
  network 172.16.0.0 0.0.255.255 area 0

! Association du protocol ospf à l'interface client
router ospf 2 vrf [Nom VRF client]
  network [Réseau interne client] [Whildcard interne client] area [Numéro de zone client]
  redistribute bgp 100 subnets
  ! En cas de propagation de route par défaut, ajouter la ligne suivante
  default-information originate

! Mise en place de BGP pour l'AS client
router bgp [Numéro AS client]
  neighbor [IP routeur client voisin] remote-as [Numéro AS client]
  neighbor [IP routeur client voisin] update-source Loopback 0
  address-family vpnv4
    neighbor [IP routeur client voisin] activate
    neighbor [IP routeur client voisin] send-community both
  address-family ipv4 vrf [Nom VRF client]
    redistribute ospf 2 vrf [Nom VRF client]
\end{lstlisting}

\seefile{nuage-supra/NG-1.cfg}

\section{Modèles de configuration \tlr}

Le réseau de \tlr est tres étendu et possède plusieurs types de configuration possible en fonction du rôle du routeur.
Ces modèles sont décris ci-apres.

\subsection{Routeur "Front" d'accès au nuage}

Les routeurs dits \emph{"Front"} sont les routeurs directement connectés au nuage.
Ces routeurs servent d'interface entre le nuage et le réseau du site.
Ils sont configurés pour la redistribution des routes du nuage (distribués en OSPF) vers les routes du site et son protocol de routage spécifique.

\begin{lstlisting}[caption=Modèle de configuration d'un routeur "Front"]
enable
conf t

!Nom unique du routeur sur le réseau Telredor
hostname [Nom routeur]

! Liaison avec le routeur d'accès nuagique du fourniseeur d'acces
interface Serial 0/0
  encapsulation ppp
  ip address [IP de liaison] 255.255.255.252
  no shutdown

! Liaison avec un autre routeur du site
interface FastEthernet 0/1
  ip address [IP du routeur] 255.255.255.252
  no shutdown

! Connexion OSPF avec le routeur de liaison
router ospf 2
  network [Réseau site] [Whildcard site] area [Numéro de zone site]
  ! Résumé de la plage d'adresse du site
  area [Numéro de zone site] range [Réseau site] [Masque site]
\end{lstlisting}

\seefile{unite-production/UP-Front.cfg}

Chaque site (excepté l'usine de production) dispose d'un protocol de routage spécifique.
Pour s'adapter a chaque routage, il est possible d'ajouter le modèle suivant pour un routeur \emph{"Front"}.

\subsubsection{Routage EIGRP}

\begin{lstlisting}[caption=Configuration d'un routeur "Front" avec EIGRP]
! Configuration d'un routage EIGRP avec redistribution depuis OSPF
router eigrp 1
  network [Réseau site] [Wildcard site]
  no auto-summary
  redistribute ospf 2

! Redirection des routes EIGRP vers OSPF
router ospf 2
  redistribute eigrp 1 subnets
  ! Dans le cas d'une propagation de route par défaut
  default-information originate
\end{lstlisting}

\seefile{unite-stockage/US-Front.cfg}

\subsubsection{Routage Statique}

\begin{lstlisting}[caption=Configuration d'un routeur "Front" avec routage statique]
! Ajout d'une route statique pour chaque routeur du site
ip route [Réseau cible] [Masque cible] [Adresse du voisin]

! Redirection des routes statiques vers OSPF
router ospf 2
  redistribute static subnets
\end{lstlisting}

\seefile{direction-general/DG-Front.cfg}

\subsubsection{Routage RIPv2}

\begin{lstlisting}[caption=Configuration d'un routeur "Front" avec RIPv2]
! Configuration d'un routage RIPv2 avec redistribution depuis OSPF
router rip
  version 2
  network [Réseau du site]
  redistribute ospf 2

! Redirection des routes RIPv2 vers OSPF
router ospf 2
  redistribute rip subnets
\end{lstlisting}

\seefile{agence-commerciale/AC-Front.cfg}

\subsection{Routeur de site}

Chaque site peux disposer d'un ou plusieurs routeurs relais permettant un routage en interne du site.
Comme chaque site dispose d'un protocole de routage différent, il est nécéssaire d'avoir une configuration spécifique pour chaque type de routage.

\begin{lstlisting}[caption=Configuration de base d'un routeur de site]
enable
conf t

!Nom unique du routeur sur le réseau Telredor
hostname [Nom routeur]

! Liaison vers un autre routeur
interface FastEthernet 0/1
  ip address [Adresse Point-à-point] 255.255.255.252
  no shutdown
\end{lstlisting}

\subsubsection{Routage OSPF}

\begin{lstlisting}[caption=Configuration d'un routeur de site avec OSPF]
! Configuration d'un routage OSPF
router ospf 2
  network [Réseau site] [Whildcard site] area [Numéro de zone site]
  ! Résumé de la plage d'adresse du site
  area [Numéro de zone site] range [Réseau site] [Masque site]
\end{lstlisting}

\seefile{unite-production/UP-1.cfg}

\subsubsection{Routage EIGRP}

\begin{lstlisting}[caption=Configuration d'un routeur de site avec EIGRP]
! Configuration d'un routage EIGRP
router eigrp 1
  network [Réseau site] [Wildcard site]
  no auto-summary
\end{lstlisting}

\seefile{unite-stockage/US-1.cfg}

\subsubsection{Routage Statique}

\begin{lstlisting}[caption=Configuration d'un routeur de site avec routage statique]
! Ajout d'une route statique pour chaque routeur du site
ip route [Réseau cible] [Masque cible] [Adresse du voisin]
\end{lstlisting}

\seefile{direction-general/DG-1.cfg}

\subsubsection{Routage RIPv2}

\begin{lstlisting}[caption=Configuration d'un routeur de site avec RIPv2]
! Configuration d'un routage RIPv2
router rip
  version 2
  network [Réseau du site]
  redistribute ospf 2
\end{lstlisting}

\seefile{agence-commerciale/AC-1.cfg}

\subsection{Bureau de proximité}

Les bureaux de proximité sont des sites completement séparés du reste du réseau \tlr.
Ils disposent d'un unique routeur faisant office de DHCP pour l'ensemble du site.
Il accede à internet au travers d'un NAT.

\begin{lstlisting}[caption=Configuration d'un routeur de bureau de proximité]
enable
conf t

!Nom unique du routeur sur le réseau du Bureau de proximité
hostname [Nom routeur]

! Configuration de l'interface local a laquelle l'ensemble des postes
! du bureaux seront connectés
interface FastEthernet 0/0
  ip address [IP interne bureau] [Masque interne bureau]
  ip nat inside
  no shutdown

! Configuration de l'interface intenet
interface Serial 0/0
  ip address [IP externe bureau] [Masque externe bureau]
  encapsulation ppp
  ip nat outside
  no shutdown

! Définition de la route par défaut vers internet
ip route 0.0.0.0 0.0.0.0 Serial 0/0

! Définition des routes pouvant être traduite via NAT
access-list 1 permit 192.168.0.0 0.0.255.255
! Configuration d'un nat en mode "surcharge" sur l'interface exposée
ip nat inside source list 1 interface Serial 0/0 overload

! Exclusion de l'adresses du routeur de la plage DHCP
ip dhcp excluded-address [IP interne bureau] [IP interne bureau]

!Configuration d'un DHCP avec DNS externe et un bail d'un jour
ip dhcp pool bureau-telredor
  network [Réseau interne bureau] [Masque interne bureau]
  default-router [IP interne bureau]
  dns-server 1.1.1.1
  lease 1
\end{lstlisting}

\seefile{BP-Front.cfg}

\subsection{Accès à internet}

L'acces à internet est mis en place sur 2 routeurs dans la topologie \tlr.
Cet acces a mis en place par un NAT en mode "surcharge".
On retrouve ce modèle de configuration sur le routeur d'acces et le routeur \emph{"Front"} de la direction général.

\begin{lstlisting}[caption=Configuration d'un accès internet avec NAT]
! Définition de l'interface interne comme "inside"
interface Serial 0/0
  ip nat inside

! Configuration de l'interface externe
interface Serial 0/1
  encapsulation ppp
  ip nat outside
  ip address [IP externe] [Masque externe]
  no shutdown

! Configuration de la route par défaut vers l'interface externe
ip route 0.0.0.0 0.0.0.0 Serial 0/1

! Configuration du routage ospf et de la distribution de la route par défaut
router ospf 2
  network [Réseau interne] [Wildcard interne] area [Numéro de zone ospf]
  redistribute static subnets
  default-information originate

! Définition des routes pouvant être traduite via NAT
access-list 1 permit 10.0.0.0 0.255.255.255
! Configuration d'un nat en mode "surcharge" sur l'interface exposée
ip nat inside source list 1 interface Serial 0/1 overload
\end{lstlisting}

\seefile{ACC-IN.cfg}
\seefile{DG-Front.cfg}

\section{Plan de reprise d'activité}

Le plan de reprise d'activité (aussi appelé PRA) est un ensemble de protocols à appliquer en cas de problème sur le réseau.
Plusieurs scénario ont été prévus et une procédure est données pour guider les intervenant à corriger au plus vite le problème et retrouver un réseau stable le plus vite possible.

\subsection{Rupture du lien avec la direction générale}

En cas de rupture du lien de la direction générale au reste du réseau, les sites ayant besoin d'une liaison urgente avec la direction général peuvent s'y connecter au travers d'une liaison VPN originellement dédié aux nomades.

\bigskip

\begin{enumerate}
\item Mettre en place un tunnel sur chaque routeur \emph{"Front"} vers la direction général pour les transmission les plus urgentes
\item Utiliser un routeur spare pour re-connecter la direction général au nuage
\item Vérifier que les sites peuvent acceder à la direction général par ce routeur de secours
\item Couper les liens VPN
\item Analyser et corriger le problème initial
\item Remettre en place le routeur initial pour reprendre le traffic initial
\end{enumerate}

\subsection{Rupture du routeur d'accès principal}

En cas de rupture du routeur d'accès principal, l'ensemble du réseau \tlr sera impacté et aucun traffic ne sera possible vers internet excepté la direction général.

\bigskip

\begin{enumerate}
  \item Définir une nouvelle route par défaut vers le routeur de la direction générale pour redonner accès à l'ensemble des sites au réseau internet
  \item Mettre en place un routeur spare d'accès pour libèrer le routeur de la direction générale.
  \item Analyser et corriger le problème initial
  \item Remplacer le routeur spare par le routeur corrigé et fonctionnel
  \item Définir la route par défaut vers le routeur initial pour reprendre un traffic standard
\end{enumerate}

\subsection{Infection d'un site par un logiciel malveillant}

En cas d'infection d'un site par un logiciel malveillant, il est possible que l'infection se propage sur le réseau et il faut limiter les dégats en coupant la connection direct à tout les différents éléments du réseau.

\bigskip

\begin{enumerate}
  \item Couper la connexion nuagique de tout les sites
  \item Couper tout les liens VPN
  \item Mettre en place la connexion internet spare de chaque site pour conserver un accès à internet
  \item Analyser et nettoyer les machines source de l'infection
  \item Analyser et nettoyer les machines critiques du réseau (Serveurs, Dirigeants, etc.)
  \item Analyser et nettoyer les autres machines du réseau
  \item Reconnecter progressivement les différents sites et vérifier la non-prolifération de la menace
\end{enumerate}

\subsection{Rupture du lien au nuage}

En cas de rupture du lien au nuage \spr, il faut se détacher de l'ensemble du nuage et utiliser les liens internet spare pour établir des tunnels VPN durant le temps de réparation.

\bigskip

\begin{enumerate}
  \item Mettre en place la connexion internet spare sur l'ensemble des sites
  \item Établir un tunnel VPN en direction de la direction générale
  \item Couper les liens au nuage de l'ensemble des sites
  \item Utiliser les tunnels VPN pour les futurs échanges
  \item Analyser et corriger le problème si possible
  \item Contacter \spr pour une correction en interne
  \item Apres la remise en place du nuage, rétablir le traffic standard
\end{enumerate}

\end{document}